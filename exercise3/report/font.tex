\subsection{Using fonts}

Flashing files

Trying bmp

We decided to use the Portable bitmap format (.pbm).
This is a much simpler format than the normal bitmap format (.bmp),
yet offers everything we need.
Files in the .pbm format have some header data first, then a sequence of 1s and 0s that represent a simple monochrome image.
We procured some pixel fonts in .bmp format online, then converted them to .pbm and included with the flash.

First we needed to read the pbm file, parse the data and store it to memory.
We created a simple struct to hold the data.

\noindent{
\begin{minipage}{\linewidth}
\begin{lstlisting}[language=C, label=pbm_image_t, caption=Portable bitmap format data struct]
typedef struct {
    unsigned int x;
    unsigned int y;
    bool* data;
} pbm_image_t;

\end{lstlisting}
\end{minipage}
}


Refer to the attached source code for the \texttt{path\_to\_pbm} function in the file \texttt{font.c}.
It is a bit too long to include in this report, but what it does is simply read a .pbm file and parse the sequence of 1s and 0s into the boolean array \texttt{pbm\_image\_t->data}.

Next up was to separate the data we now had in memory into individual boolean arrays representing individual character masks.
Again we created a simple struct that could hold the boolean data for the character mask.

\noindent{
\begin{minipage}{\linewidth}
\begin{lstlisting}[language=C, label=char_t, caption=Character mask struct]
typedef struct {
    char name;
    bool* data;
} char_t;
\end{lstlisting}
\end{minipage}
}


In addition we created another struct to hold the collection of different character masks (a font).

\noindent{
\begin{minipage}{\linewidth}
\begin{lstlisting}[language=C, label=font_t, caption=Font struct]
typedef struct {
    unsigned int char_w;
    unsigned int char_h;
    char_t* chars;
} font_t;
\end{lstlisting}
\end{minipage}
}


% glyph stuff in next subsection
