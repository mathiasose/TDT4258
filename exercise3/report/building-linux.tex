\section{Configuring and building Linux}

As a first step in this exercise, we configured PTXdist and flashed all images to the development board, as outlined in section 5.3 in the compendium \cite[p.~47]{compendium} and section \ref{ptxdist-usage} above. This part of the exercise was fairly straight forward. See summary of steps in listing \ref{setup-summary}

\begin{lstlisting}[label=setup-summary, caption=Setup summary]
> ptxdist select configs/ptxconfig
> ptxdist platform configs/platform-energymicro-efm32gg-dk3750/platformconfig
> ptxdist toolchain /opt/ex3/OSELAS.Toolchain-2012.12.0/arm-cortexm3-uclinuxeabi/gcc-4.7.2-uclibc-0.9.33.2-binutils-2.22-kernel-3.6-sanitized/bin
> ptxdist images
> ptxdist test flash-all
\end{lstlisting}

Having done this, we started \texttt{miniterm.py} and started exploring. One of the first things we tried improving was the general development workflow. 
We tried amending a \texttt{printk} statement in the driver, to familiarize ourselves with the minimal set of steps necessary to get new code onto the board during development. 
The steps in listing \ref{devel-steps} outline what has to be done to get any changes in the source for a package running on the development board. This is quite tedious, so we made a project wide Makefile simplifying the process. (Listing \ref{dev-makefile})

\begin{lstlisting}[label=devel-steps, caption=Development steps]
> ptxdist clean <name_of_package>
> ptxdist compile <name_of_package>
> ptxdist image root.romfs
> ptxdist test flash-rootfs
\end{lstlisting}

\begin{lstlisting}[language=make, label=dev-makefile, caption=Development Makefile]
.PHONY: init build

init:
    ptxdist images
    ptxdist test flash-all

build:
    ptxdist clean game
    ptxdist clean driver-gamepad
    ptxdist go
    ptxdist image root.romfs
    ptxdist test flash-rootfs
\end{lstlisting}
