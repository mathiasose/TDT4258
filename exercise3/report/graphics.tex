\subsection{Using the development board LED display}
As advised in the compendium, we used memory mapping to manipulate the image to be displayed.

\noindent{
\begin{minipage}{\linewidth}
\begin{lstlisting}[language=C, label=init_framebuffer, caption=Framebuffer initializing function]
int init_framebuffer()
{
    fbfd = open("/dev/fb0", O_RDWR);
    if (fbfd == -1) {
        printf("Error: unable to open framebuffer device.\n");
        return EXIT_FAILURE;
    }

    // Get screen size info
    if (ioctl(fbfd, FBIOGET_VSCREENINFO, &vinfo) == -1) {
        printf("Error: unable to get screen info.\n");
        return EXIT_FAILURE;
    }

    screensize_pixels = vinfo.xres * vinfo.yres;
    screensize_bytes = screensize_pixels * vinfo.bits_per_pixel / 8;

    fbp = (uint16_t*)mmap(NULL, screensize_bytes, PROT_READ | PROT_WRITE, MAP_SHARED, fbfd, 0);
    if ((int)fbp == MAP_FAILED) {
        printf("Error: failed to memorymap framebuffer.\n");
        return EXIT_FAILURE;
    }

    return EXIT_SUCCESS;
}
\end{lstlisting}
\end{minipage}
}


This maps a region of memory, identified by the pointer \texttt{fbp}, to \texttt{/dev/fb0}, the interface for manipulating the development board LED display.
Each 16-bit integer in this region represents the RGB565 color of a single pixel.
With this we were now able to draw shapes by iterating over the memory region and writing different values.
After changing pixels, the function \texttt{ioctl(fbfd, FB\_DRAW, \&screen)} is called to tell the screen to refresh the pixels in the \texttt{screen} area.

One thing we noticed and were confused by for quite some time was a black square region of ~5-10 pixels that would not go away.
It was eventually revealed to us that this was a cursor that uCLinux placed on the screen.
We were also told that this could be removed by running the command \texttt{echo 0 > /sys/class/graphics/fbcon/cursor\_blink} on the board.
