\chapter*{Abstract}


In this exercise, the group expanded upon the knowledge acquired in Exercise 1 \cite{exercise1report} and learned about timer-based interrupts and generating audio using a DAC. As in the previous exercise, an ARM compatible GNU toolchain was used to program the EFM32GG development board. Both group members learned a lot about using the C language to program microcontrollers as well as about peripherals on the development board we had not previously used. To comply with exercise requirements \cite[p.~42]{compendium}, we implemented three different sound effects and three different songs/melodies. We also implemented several abstraction layers above simply writing samples to the DAC, allowing us to ``compose'' new songs by listing individual notes, i.e. \texttt{C4 E4 G4 C5 E5 G5} and defining how long each note should be played. With respect to energy efficiency, we utilized the EFM32GGs energy saving modes to achieve low energy consumption whilst not playing songs.
