\section{Project setup and toolchain description}

After downloading and untarring the support files for this exercise, we chose to reorganize the project structure slightly. We decided upon the directory layout in listing \ref{lst:directory-layout}

\begin{lstlisting}[label=lst:directory-layout,caption=Directory layout]
exercise1
|-- Makefile
|-- NOTES
|-- efm32gg.ld
|-- lib
|   |-- efm32gg.s
|   |-- vector.s
|-- report
|   |-- \ldots
|-- src
    |-- main.s
\end{lstlisting}

Following this, we made the necessary edits to the Makefile, updating all rules with the new directory layout. As can be seen from listing \ref{lst:directory-layout} we added a \texttt{vector.s} file to the lib directory. This contains the vector table defined at the top of the Assembly file supplied by the subject staff. In addition to the directories listed above, we also added \texttt{build/} and \texttt{exe/} directories, to contain build artefacts.

\subsection{GNU Toolchain \& GNU Make}

Throughout the exercise a version of the GNU toolchain made especially for cross development on microcontrollers based on ARM embedded processors was used. The toolchain was preinstalled on the lab computers, but is freely available online if one wishes to install it on a personal computer. The following is a brief description of how we used each of the tools in this exercise.

\subsubsection{GNU AS}

To assemble our program code to object files we used an ARM-specific version of the GNU AS assembler, a general example of usage is shown in listing \ref{lst:gnu-as-ex}

\begin{lstlisting}[label=lst:gnu-as-ex,caption=Assembler usage]
arm-none-eabi-as -mcpu=cortex-m3 -mthumb -g -o <output.o> <source.s>
\end{lstlisting}

The result is a GDB-debuggable object file with code conforming to the Cortex-M3 instruction set based on the assembly code in the source file.

\subsubsection{GNU LD}

To link the object files assembled by AS into an executable, an ARM version of GNU LD was used. Syntax in listing \ref{lst:gnu-ld-ex}

\begin{lstlisting}[label=lst:gnu-ld-ex,caption=Linker usage, mathescape]
arm-none-eabi-ld -T <linkerscript.ld> -nostdlib -o <output.elf>
    $\hookrightarrow$ <arg0.o> <arg1.o> <arg2.o>
\end{lstlisting}

The linkerscript supplied defines how the memory on the microcontroller is to be used.

\subsubsection{GNU Objcopy}

To upload the executable to the development board, we need a clean binary file, void of metadata. This is achieved with the GNU objcopy utility. Syntax is as follows:

\begin{lstlisting}[label=lst:gnu-objcopy-ex,caption=Objcopy usage]
arm-none-eabi-objcopy -j .text -O binary <input.elf> <output.bin>
\end{lstlisting}


\subsubsection{GNU Make}

Having to manually assemble, link and copy every time we want to test a modification in the code quickly becomes tedious. Thankfully, in the handout files a Makefile is included including build rules automating all these tasks, allowing us to use the command \texttt{make} as an end-to-end solution for creating new binary files. An \texttt{upload} rule was also included. We chose to combine the two rules into a single \texttt{run} rule, 
