\chapter{Introduction}
In this exercise we take a very hands on approach to learning microcontroller programming in Assembly. This should give us some valuable knowledge about the inner workings of the microcontroller, and be a good foundation for doing more advanced things in later exercises.

The device used in this exercise is called EFM32GG-DK3750. It is produced by Silicon Labs for developing and testing programs for embedded systems running ARM processors. Programs are written and compiled on a personal computer then flashed to the development board via USB. The development board has multiple I/O capabilities, but in this exercise we only utilize the GPIO pins. The development board also features a built in screen that can plot current spent by the microcontroller and allows some settings to be edited live. Amperage can also be monitored on the connected PC using provided software.

The primary objective of the exercise is to use a gamepad peripheral as an input device, send input to the device using it's GPIO functionality, handle the input, then send some sort of output based on the input back through some other GPIO pins to the gamepad where they will activate some LEDs.

A secondary but important objective of the exercise is to observe the energy efficiency of the system when running the program, and see what measures can be taken to reduce it. Using interrupt-based handling instead of continuous polling is especially useful to reduce power consumption, as it allows the microcontroller to sleep until input happens, drastically reducing the consumed power. Both methods will be tried and results will be analyzed.
