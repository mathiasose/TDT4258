\chapter{Introduction}
In this exercise we take a very hands on approach to learning microcontroller programming in Assembly. This should give us some valuable knowledge about the inner workings of the microcontroller, and be a good foundation for doing more advanced things in later exercises.
The EFM32GG-DK3750 device has an ARM processor to which a program can be flashed via USB. It also has multiple GPIO pins. These are the features that will mainly be used in this exercise.
The objective of the exercise is to use a gamepad peripheral as input device, send input to the device using it's GPIO functionality, handle the input and send some sort of output based on the input back through some other GPIO pins to the gamepad where they will activate some LEDs.
Another objective of the exercise is to observe the power consumption of the microcontroller when running the program, and see what measures can be taken to reduce it. Using interupt-based handling instead of continuous polling is especially useful to achieve this, as it allows the microcontroller to sleep until input happens, drastically reducing power consumption. Both methods will be tried so that they can be compared using the EFM32GG's built in power-monitoring feature.
