\subsection{Bare essentials}
\label{sec:bare-essentials}

In the spirit of developing iteratively, our first goal was to create a minimal, compiling program that we could flash to the development board and test our workflow, described in chapter \ref{chap:description}. With the modifications we made to the project layout, such a minimal program would look something like this:

\begin{lstlisting}[label=minimal-program-ex,caption=A minimal program]
.syntax unified
.include "lib/efm32gg.s"
.include "lib/vector.s"

.section .text
// Reset handler
.globl _reset
.type _reset, %function
.thumb_func
_reset:
    // Aliases
    W .req R3
    GPIO_O .req R4
    GPIO_I .req R5
    GPIO .req R6
    T0 .req R7
    T1 .req R8
    T2 .req R9

    // Load GPIO base addresses into the relevant registers
    LDR R4, =GPIO_PA_BASE
    LDR R5, =GPIO_PC_BASE
    LDR R6, =GPIO_BASE

// GPIO_Handler
.thumb_func
gpio_handler:
    B .

// Dummy handler
.thumb_func
dummy_handler:
    B .
\end{lstlisting}

At this point, there was still some confusion as to how one should start a GDB session interacting with the program running on the development board. The compendium originally suggested a nonexistent \texttt{gdbserver.sh} script, and the error was not corrected until we had moved on to the next phase.
