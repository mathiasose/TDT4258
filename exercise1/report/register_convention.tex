\subsection{Register Convention}

In an early stage of development it became apparent that the limited number of registers available posed a challenge for programmers used to having unlimited variables available in higher level languages.
After investigating how the ARM register convention was defined, a new convention was defined on top of the ARM convention to fit our purposes, and some aliases were added to the program to make the convention easier to use.

\begin{table}[h!]
    \label{tbl:register-convention} 
    \begin{tabular}{l | l | l}
        Register        & Alias             & Description                                            \\
        \hline
        \texttt{R0}     & ~                 & Reserved for subroutine argument by ARM convention     \\
        \texttt{R1}     & ~                 & Reserved for subroutine argument by ARM convention     \\
        \texttt{R2}     & ~                 & Reserved for subroutine argument by ARM convention     \\
        \texttt{R3}     & \texttt{W}        & Reserved for subroutine argument by ARM convention     \\
        ~               & ~                 & Used for the countdown for the wait subroutine.        \\
        \texttt{R4}     & \texttt{GPIO\_O}  & Reserved for addressing GPIO\_PA\_BASE (LED outputs)   \\
        \texttt{R5}     & \texttt{GPIO\_I}  & Reserved for addressing GPIO\_PC\_BASE (button inputs) \\
        \texttt{R6}     & \texttt{GPIO}     & Reserved for addressing GPIO\_BASE                     \\
        \texttt{R7}     & \texttt{T0}       & Temporary variable                                     \\
        \texttt{R8}     & \texttt{T1}       & Temporary variable                                     \\
        \texttt{R9}     & \texttt{T2}       & Temporary variable                                     \\
        \texttt{R10}    & ~                 & Unused                                                 \\
        \texttt{R11}    & ~                 & Unused                                                 \\
        \texttt{R12}    & \texttt{IP}       & Reserved for Intra-Procedure-call by ARM convention    \\
        \texttt{R13}    & \texttt{SP}       & Reserved for Stack Pointer by ARM convention           \\
        \texttt{R14}    & \texttt{LR}       & Reserved for Link Register  by ARM convention          \\
        \texttt{R15}    & \texttt{PC}       & Reserved for Program Counter by ARM convention         \\
    \end{tabular}
    \caption{Register convention}
\end{table}

We implemented this using the \texttt{.req} directive. The syntax is as follows:

\begin{lstlisting}[label=register-aliasing,caption=Register aliasing]
name .req register
T0 .req R7
\end{lstlisting}

