\subsection{Register Convention}

In an early stage of development it became apparent that the limited number of registers available posed a challenge for programmers used to having unlimited variables available in higher level languages.
After investigating how the AMR register convention was defined, a new convention was defined on top of the ARM convention to fit our purposes, and some aliases were added to the program to make the convention easier to use.

\begin{table}
    \begin{tabular}{lll}
        Register & Alias   & Description                                            \\
        R0       & ~       & Reserved for subroutine argument by ARM convention     \\
        R1       & ~       & Reserved for subroutine argument by ARM convention     \\
        R2       & ~       & Reserved for subroutine argument by ARM convention     \\
        R3       & W       & Reserved for subroutine argument by ARM convention.
                             Used for the countdown for the wait subroutine.        \\
        R4       & GPIO\_O & Reserved for addressing GPIO\_PA\_BASE (LED outputs)   \\
        R5       & GPIO\_I & Reserved for addressing GPIO\_PC\_BASE (button inputs) \\
        R6       & GPIO    & Reserved for addressing GPIO\_BASE                     \\
        R7       & T0      & Temporary variable                                     \\
        R8       & T1      & Temporary variable                                     \\
        R9       & T2      & Temporary variable                                     \\
        R10      & ~       & Unused                                                 \\
        R11      & ~       & Unused                                                 \\
        R12      & IP      & Reserved for Intra-Procedure-call by ARM convention    \\
        R13      & SP      & Reserved for Stack Pointer by ARM convention           \\
        R14      & LR      & Reserved for Link Register  by ARM convention          \\
        R15      & PC      & Reserved for Program Counter by ARM convention    \\
    \end{tabular}
\end{table}

We implemented this using the \texttt{.req} directive. The syntax is as follows:

\begin{lstlisting}[label=register-aliasing,caption=Register aliasing]
name .req register
T0 .req R7
\end{lstlisting}

