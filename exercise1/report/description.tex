\chapter{Description and Methodology}

The exercise was primarily done in the computer lab (room 458 at IT-Vest, NTNU). The source code was put in an online git repository allowing some minor work to be done off-site before it could be tested in the lab.
The lab had already been set up with multiple workstations with a stationary computer running Ubuntu 12.04 LTS with the neccessary software already installed, connected via USB to an EFM32GG-DK3750 development kit.
The exercise description was made available in a compendium also containing instructions about how to write and run the program. Also described was how to use the GNU Debugger to troubleshoot. A student assistant was also available for some hours every week to help.
The program itself was developed iteratively, with lots of trial and error.

In an early stage of development it became apparent that the limited number of registers available posed a challenge for programmers used to having unlimited variables available in higher level languages.
After investigating how the AMR register convention was defined, a new convention was defined on top of the ARM convention to fit our purposes, and some aliases were added to the program to make the convention easier to use.

\begin{table}
    \begin{tabular}{lll}
        Register & Alias   & Description                                            \\
        R0       & ~       & Reserved for subroutine argument by ARM convention     \\
        R1       & ~       & Reserved for subroutine argument by ARM convention     \\
        R2       & ~       & Reserved for subroutine argument by ARM convention     \\
        R3       & W       & Reserved for subroutine argument by ARM convention.
                             Used for the countdown for the wait subroutine.        \\
        R4       & GPIO\_O & Reserved for addressing GPIO\_PA\_BASE (LED outputs)   \\
        R5       & GPIO\_I & Reserved for addressing GPIO\_PC\_BASE (button inputs) \\
        R6       & GPIO    & Reserved for addressing GPIO\_BASE                     \\
        R7       & T0      & Temporary variable                                     \\
        R8       & T1      & Temporary variable                                     \\
        R9       & T2      & Temporary variable                                     \\
        R10      & ~       & Unused                                                 \\
        R11      & ~       & Unused                                                 \\
        R12      & IP      & Reserved for Intra-Procedure-call by ARM convention    \\
        R13      & SP      & Reserved for Stack Pointer by ARM convention           \\
        R14      & LR      & Reserved for Link Register  by ARM convention          \\
        R15      & PC      & Reserved for Intra-Procedure-call by ARM convention    \\
    \end{tabular}
\end{table}
